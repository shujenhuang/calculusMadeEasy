\documentclass{ximera}

\title{Finding Some Solutions}


\begin{document}
\begin{abstract}
\end{abstract}
\maketitle

In this chapter we go to work finding solutions to
some important differential equations, using for this
purpose the processes shown in the preceding chapters.

The beginner, who now knows how easy most of
those processes are in themselves, will here begin to
realize that integration is an art. As in all arts, so
in this, facility can be acquired only by diligent and
regular practice. He who would attain that facility
must work out examples, and more examples, and yet
more examples, such as are found abundantly in all
the regular treatises on the Calculus. Our purpose
here must be to afford the briefest introduction to
serious work.

Example (1.) Find the solution of the differential
equation
\[
ay + b \frac{dy}{dx} = 0.
\]

Transposing we have
\[
b \frac{dy}{dx} = -ay.
\]


Now the mere inspection of this relation tells us
that we have got to do with a case in which $\dfrac{dy}{dx}$ is
proportional to $y$. If we think of the curve which
will represent $y$ as a function of $x$, it will be such
that its slope at any point will be proportional to
the ordinate at that point, and will be a negative
slope if $y$ is positive. So obviously the curve will
be a die-away curve, and the solution will
contain $\epsilon^{-x}$ as a factor. But, without presuming on
this bit of sagacity, let us go to work.

As both $y$ and $dy$ occur in the equation and on
opposite sides, we can do nothing until we get both
$y$ and $dy$ to one side, and $dx$ to the other. To do
this, we must split our usually inseparable companions
$dy$ and $dx$ from one another.
\[
\frac{dy}{y} = - \frac{a}{b}\, dx.
\]

Having done the deed, we now can see that both
sides have got into a shape that is integrable, because
we recognize $\dfrac{dy}{y}$, or $\dfrac{1}{y}\, dy$, as a differential that we
have met with (here) when differentiating logarithms.
So we may at once write down the instructions to
integrate,
\[
\int \frac{dy}{y} = \int -\frac{a}{b}\, dx;
\]
and doing the two integrations, we have:
\[
\log_\epsilon y = -\frac{a}{b} x + \log_\epsilon C,
\]

where $\log_\epsilon C$ is the yet undetermined constant
of
integration. Then, delogarizing, we get:
\[
y = C \epsilon^{-\frac{a}{b} x},
\]
which is the solution required*. Now, this solution
looks quite unlike the original differential equation
from which it was constructed: yet to an expert
mathematician they both convey the same information
as to the way in which $y$ depends on $x$.

We may write down any form of constant as the “constant of
  integration,” and the form $\log_\epsilon C$ is adopted here by preference,
  because the other terms in this line of equation are, or are treated
  as logarithms; and it saves complications afterward if the added
  constant be of the same kind.

Now, as to the $C$, its meaning depends on the
initial value of $y$. For if we put $x = 0$ in order to
see what value $y$ then has, we find that this makes
$y = C \epsilon^{-0}$; and as $\epsilon^{-0} = 1$ we see that $C$ is nothing else
than the particular value*
 of $y$ at starting. This we
may call $y_0$, and so write the solution as
\[
y = y_0 \epsilon^{-\frac{a}{b} x}.
\]
Compare what was said about the ``constant of integration,''
  with reference to Figure 48 on here, and Figure 51.


Example (2.)

Let us take as an example to solve
\[
ay + b \frac{dy}{dx} = g,
\]
where $g$ is a constant. Again, inspecting the equation
will suggest, (1) that somehow or other $\epsilon^x$ will come
into the solution, and (2) that if at any part of the

curve $y$ becomes either a maximum or a minimum, so
that $\dfrac{dy}{dx} = 0$, then $y$ will have the value $= \dfrac{g}{a}$. But let
us go to work as before, separating the differentials
and trying to transform the thing into some integrable
shape.
\begin{align*}
b\frac{dy}{dx}           &= g -ay; \\
\frac{dy}{dx}            &= \frac{a}{b}\left(\frac{g}{a}-y\right); \\
\frac{dy}{y-\dfrac{g}{a}} &= -\frac{a}{b}\, dx.
\end{align*}

Now we have done our best to get nothing but $y$ and $dy$
on one side, and nothing but $dx$ on the other.
But is the result on the left side integrable?

It is of the same form as the result on here; so,
writing the instructions to integrate, we have:
\[
\int{\frac{dy}{y-\dfrac{g}{a}}} = - \int{\frac{a}{b}\, dx};
\]
and, doing the integration, and adding the appropriate
constant,
\begin{align*}
\log_\epsilon\left(y-\frac{g}{a}\right) &= -\frac{a}{b}x + \log_\epsilon C; \\
 \text{whence}\;\;        y-\frac{g}{a} &= C\epsilon^{-\frac{a}{b}x}; \\
 \text{and finally,}\;\;              y &= \frac{g}{a} + C\epsilon^{-\frac{a}{b}x},
\end{align*}
which is the solution.


If the condition is laid down that $y = 0$ when $x = 0$
we can find $C$; for then the exponential becomes $= 1$;
and we have
\begin{align*}
                0 &= \frac{g}{a} + C, \\
 \text{or}\;  C &= -\frac{g}{a}.
\end{align*}

Putting in this value, the solution becomes
\[
y = \frac{g}{a} (1-\epsilon^{-\frac{a}{b} x}).
\]

But further, if $x$ grows indefinitely, $y$ will grow to
a maximum; for when $x=\infty$, the exponential $= 0$,
giving $y_{\text{max.}} = \dfrac{g}{a}$. Substituting this, we get finally
\[
y = y_{\text{max.}}(1-\epsilon^{-\frac{a}{b} x}).
\]

This result is also of importance in physical science.


Example (3.)

   Let $ay+b\frac{dy}{dt} = g · \sin 2\pi nt$.


We shall find this much less tractable than the
preceding. First divide through by $b$.
\[
\frac{dy}{dt} + \frac{a}{b}y = \frac{g}{b} \sin 2\pi nt.
\]

Now, as it stands, the left side is not integrable.
But it can be made so by the artifice–and this is

where skill and practice suggest a plan–of multiplying
all the terms by $\epsilon^{\frac{a}{b} t}$, giving us:
\[
\frac{dy}{dt} \epsilon^{\frac{a}{b} t} + \frac{a}{b} y \epsilon^{\frac{a}{b} t} = \frac{g}{b} \epsilon^{\frac{a}{b} t} · \sin 2 \pi nt,
\]
which is the same as
\[
\frac{dy}{dt} \epsilon^{\frac{a}{b} t} + y \frac{d(\epsilon^{\frac{a}{b} t})}{dt} = \frac{g}{b} \epsilon^{\frac{a}{b} t} · \sin 2 \pi nt;
\]
and this being a perfect differential may be integrated
thus:–since, if  $u = y\epsilon^{\frac{a}{b} t}$, $\dfrac{du}{dt} = \dfrac{dy}{dt} \epsilon^{\frac{a}{b} t} + y \dfrac{d(\epsilon^{\frac{a}{b} t})}{dt}$,
\begin{align*}
  y \epsilon^{\frac{a}{b} t}
  &= \frac{g}{b} \int \epsilon^{\frac{a}{b} t} · \sin 2 \pi nt · dt + C, \\
 or
y &= \frac{g}{b} \epsilon^{-\frac{a}{b} t}
     \int \epsilon^{ \frac{a}{b} t} · \sin 2\pi nt · dt
       + C\epsilon^{-\frac{a}{b} t}.
\tag*{[A]}
\end{align*}

The last term is obviously a term which will die
out as $t$ increases, and may be omitted. The trouble
now comes in to find the integral that appears as a
factor. To tackle this we resort to the device (see
here) of integration by parts, the general formula for
which is $\int u dv = uv - \int v du$. For this purpose write
\begin{align*}
&\left\{
\begin{aligned}
 u &= \epsilon^{\frac{a}{b} t}; \\
dv &= \sin 2\pi nt · dt.
\end{aligned}
\right.  \\
\end{align*}
We shall then have
\begin{align*}
&\left\{
\begin{aligned}
du &= \epsilon^{\frac{a}{b} t} \times \frac{a}{b}\, dt; \\
v &= - \frac{1}{2\pi n} \cos 2\pi nt.
\end{aligned}
\right.
\end{align*}


Inserting these, the integral in question becomes:
\begin{align*}
\int \epsilon^{\frac{a}{b} t} &{} · \sin 2 \pi n t · dt \\
&= -\frac{1}{2 \pi n} · \epsilon^{\frac{a}{b} t} · \cos 2 \pi nt
   -\int -\frac{1}{2\pi n} \cos 2 \pi nt · \epsilon^{\frac{a}{b} t} · \frac{a}{b}\, dt \\
&= -\frac{1}{2 \pi n} \epsilon^{\frac{a}{b} t} \cos 2 \pi nt
   +\frac{a}{2 \pi nb} \int \epsilon^{\frac{a}{b} t} · \cos 2 \pi nt · dt.
\tag*{[B]}
\end{align*}

The last integral is still irreducible. To evade the
difficulty, repeat the integration by parts of the left
side, but treating it in the reverse way by writing:
\begin{align*}
&\left\{
\begin{aligned}
u &= \sin 2 \pi n t ; \\
dv &= \epsilon^{\frac{a}{b} t} · dt;
\end{aligned}
\right. \\[1ex]
 whence
&\left\{
  \begin{aligned}
  du &= 2 \pi n · \cos 2 \pi n t · dt; \\
 v &= \frac{b}{a} \epsilon ^{\frac{a}{b} t}
\end{aligned}
\right.
\end{align*}

Inserting these, we get
\begin{align*}
\int \epsilon^{\frac{a}{b} t} &{} · \sin 2 \pi n t · dt\\
&= \frac{b}{a} · \epsilon^{\frac{a}{b} t} · \sin 2 \pi n t -
   \frac{2 \pi n b}{a} \int \epsilon^{\frac{a}{b} t} · \cos 2 \pi n t · dt. \tag*{[C]}
\end{align*}

Noting that the final intractable integral in [C] is
the same as that in [B], we may eliminate it, by
multiplying [B] by $\dfrac{2 \pi nb}{a}$, and multiplying [C] by
$\dfrac{a}{2 \pi nb}$, and adding them.


The result, when cleared down, is:
\begin{align*}
\int \epsilon^{\frac{a}{b} t} · \sin 2 \pi n t · dt
  &= \epsilon^{\frac{a}{b} t} \left\{\frac{ ab · \sin 2 \pi nt - 2 \pi n b^2 · \cos 2 \pi n t}{ a^2 + 4 \pi^2 n^2 b^2 } \right\}
\tag*{[D]} &\\
\end{align*}
Inserting this value in [A], we get
\begin{align*}
y &= g \left\{\frac{ a · \sin 2 \pi n t - 2 \pi n b · \cos 2 \pi nt}{ a^2 + 4 \pi^2  n^2 b^2}\right\}. &
\end{align*}

To simplify still further, let us imagine an angle $\phi$
such that $\tan \phi = \dfrac{2 \pi n b}{ a}$.
   Then
\[
\sin \phi = \frac{2 \pi nb}{\sqrt{a^2 + 4 \pi^2 n^2 b^2}},
\]
 and
 \[
\cos \phi = \frac{a}{\sqrt{a^2 + 4 \pi^2 n^2 b^2}}.  \\
\]
Substituting these, we get:
 \[
y = g \frac{\cos \phi · \sin 2 \pi nt
  - \sin \phi · \cos 2 \pi nt}{\sqrt{a^2 + 4 \pi^2 n^2 b^2}}, \\
\]
which may be written
 \[
y = g \frac{\sin(2 \pi nt - \phi)}{\sqrt{a^2 + 4 \pi^2 n^2 b^2}},
\]
 which is the solution  desired.


This is indeed none other than the equation of an
alternating electric current, where $g$ represents the
amplitude of the electromotive force, $n$ the frequency,
$a$ the resistance, $b$ the coefficient of self-induction of
the circuit, and $\phi$ is an angle of lag.




Example (4.)


   Suppose that
$M\, dx + N\, dy = 0.$


We could integrate this expression directly, if $M$
were a function of $x$ only, and $N$ a function of $y$
only; but, if both $M$ and $N$ are functions that depend
on both $x$ and $y$, how are we to integrate it? Is it
itself an exact differential? That is: have $M$ and $N$
each been formed by partial differentiation from some
common function $U$, or not? If they have, then
\[\left\{
  \begin{aligned}
 \frac{\partial U}{\partial x} = M, \\
 \frac{\partial U}{\partial y} = N.
  \end{aligned}
\right.
\]
And if such a common function exists, then
\[
\frac{\partial U}{\partial x}\, dx + \frac{\partial U}{\partial y}\, dy
\]
is an exact differential (compare here).


Now the test of the matter is this. If the expression
is an exact differential, it must be true that
\begin{align*}
        \frac{dM}{dy} &= \frac{dN}{dx}; \\
\text{for then}\;
\frac{d(dU)}{dx\, dy} &= \frac{d(dU)}{dy\, dx},\\
\end{align*}
 which is necessarily true.

Take as an illustration the equation
\[
(1 + 3 xy)\, dx + x^2\, dy = 0.
\]

Is this an exact differential or not? Apply the
test.
\[\left\{
  \begin{aligned}
  \frac{d(1 + 3xy)}{dy}=3x, \\
  \dfrac{d(x^2)}{dx} = 2x,
  \end{aligned}
\right.
\]
which do not agree. Therefore, it is not an exact
differential, and the two functions $1+3xy$ and $x^2$
have not come from a common original function.

It is possible in such cases to discover, however, an
integrating factor, that is to say, a factor such that
if both are multiplied by this factor, the expression
will become an exact differential. There is no one
rule for discovering such an integrating factor; but
experience will usually suggest one. In the present
instance $2x$ will act as such. Multiplying by $2x$, we
get
\[
(2x + 6x^2y)\, dx + 2x^3\, dy = 0.
\]

Now apply the test to this.
\[
\left\{
  \begin{aligned}
 \frac{d(2x + 6x^2y)}{dy}=6x^2, \\
 \dfrac{d(2x^3)}{dx} = 6x^2,
  \end{aligned}
\right.
\]
which agrees. Hence this is an exact differential, and
may be integrated. Now, if $w = 2x^3y$,
\[
dw=6x^2y\, dx + 2x^3\, dy.
\]
   Hence
   \[
\int 6x^2y\, dx + \int 2x^3\, dy=w=2x^3y;
\]
 so that we get
\[
 U = x^2 + 2x^3y + C.
\]


Example (5.) Let $\dfrac{d^2 y}{dt^2} + n^2 y = 0$.

In this case we have a differential equation of the
second degree, in which $y$ appears in the form of
a second differential coefficient, as well as in person.

Transposing, we have $\dfrac{d^2 y}{dt^2} = - n^2 y$.

It appears from this that we have to do with a
function such that its second differential coefficient is
proportional to itself, but with reversed sign. In
Chapter XV. we found that there was such a function–namely,
the sine (or the cosine also) which
possessed this property. So, without further ado,
we may infer that the solution will be of the form
$y = A \sin (pt + q)$. However, let us go to work.

Multiply both sides of the original equation by $2\dfrac{dy}{dt}$
and integrate, giving us $2\dfrac{d^2 y}{dt^2}\, \dfrac{dy}{dt} + 2x^2 y \dfrac{dy}{dt} = 0$, and, as
\[
2 \frac{d^2y}{dt^2}\, \frac{dy}{dt}
  = \frac{d \left(\dfrac{dy}{dt}\right)^2}{dt},\quad
\left(\frac{dy}{dt}\right)^2 + n^2 (y^2-C^2) = 0,
\]
$C$ being a constant. Then, taking the square roots,
\[
\frac{dy}{dt} = -n \sqrt{ y^2 - C^2}\quad \text{and}\quad
\frac{dy}{\sqrt{C^2 - y^2}} = n · dt.
\]

But it can be shown that (see here)
\[
\frac{1}{\sqrt{C^2 - y^2}} = \frac{d (\arcsin \dfrac{y}{C})}{dy};
\]
whence, passing from angles to sines,
\[
\arcsin \frac{y}{C} = nt + C_1\quad \text{and}\quad y = C \sin (nt + C_1),
\]

where $C_1$ is a constant angle that comes in by integration.

Or, preferably, this may be written
\[
y = A \sin nt + B \cos nt, \text{ which is the solution.}
\]


Example (6.) $\dfrac{d^2 y}{dt^2} - n^2 y = 0$.

Here we have obviously to deal with a function $y$
which is such that its second differential coefficient is
proportional to itself. The only function we know
that has this property is the exponential function
(see here), and we may be certain therefore that the
solution of the equation will be of that form.

Proceeding as before, by multiplying through by
$2 \dfrac{dy}{dx}$, and integrating, we get $2\dfrac{d^2 y}{dx^2}\, \dfrac{dy}{dx} - 2x^2 y \dfrac{dy}{dx}=0$,

 and, as
\[
2\frac{d^2 y}{dx^2}\, \frac{dy}{dx}
  = \frac{d \left(\dfrac{dy}{dx}\right)^2}{dx},\quad
\left(\frac{dy}{dx}\right)^2 - n^2 (y^2 + c^2) = 0, \\
\frac{dy}{dx} - n \sqrt{y^2 + c^2} = 0,
\]
where $c$ is a constant, and $\dfrac{dy}{\sqrt{y^2 + c^2}} = n\, dx$.

Now, if
\[
\quad w = \log_\epsilon ( y+ \sqrt{y^2+ c^2}) = \log_\epsilon u,\\
\frac{dw}{du} = \frac{1}{u},\quad \frac{du}{dy} = 1 + \frac{y}{\sqrt{y^2 + c^2}} = \frac{y + \sqrt{ y^2 + c^2}}{\sqrt{y^2 + c^2}} \\
\]
 and
\[
 \frac{dw}{dy} = \frac{1}{\sqrt{ y^2 + c^2}}.
\]

Hence, integrating, this gives us
\[
\log_\epsilon (y + \sqrt{y^2 + c^2} ) = nx + \log_\epsilon C, \\
y + \sqrt{y^2 + c^2} = C \epsilon^{nx}.
\tag*{(1)}  \\
\]
\[
   \text{Now}\;  \qquad ( y + \sqrt{y^2 + c^2} ) \times ( -y + \sqrt{y^2 + c^2} ) = c^2 ;    \\
  \text{whence}\;  \qquad  -y + \sqrt{y^2 + c^2} = \dfrac{c^2}{C} \epsilon^{-nx}.
\tag*{(2)}
\]


Subtracting (2) from (1) and dividing by $2$, we
then have
\[
y = \frac{1}{2} C \epsilon^{nx} - \frac{1}{2}\, \frac{c^2}{C} \epsilon^{-nx},
\]
which is more conveniently written
\[
y = A \epsilon^{nx} + B \epsilon^{-nx}.
\]
Or, the solution, which at first sight does not look
as if it had anything to do with the original equation,
shows that $y$ consists of two terms, one of which
grows logarithmically as $x$ increases, and of a second
term which dies away as $x$ increases.


Example (7.)
   Let
\begin{align*}
b \frac{d^2y}{dt^2} + a \frac{dy}{dt} + gy &= 0.
\end{align*}

Examination of this expression will show that, if
$b = 0$, it has the form of Example 1, the solution of
which was a negative exponential. On the other
hand, if $a = 0$, its form becomes the same as that of
Example 6, the solution of which is the sum of a
positive and a negative exponential. It is therefore
not very surprising to find that the solution of the
present example is
\begin{align*}
y &= (\epsilon^{-mt})(A \epsilon^{nt} + B \epsilon^{-nt}), \\
 \text{where}\;
m &= \frac{a}{2b}\quad \text{and}\quad
n  = \sqrt{\frac{a^2}{4b^2}} - \frac{g}{b}.
\end{align*}

The steps by which this solution is reached are not
given here; they may be found in advanced treatises.


Example (8.)

\[
\frac{d^2y}{dt^2} = a^2 \frac{d^2y}{dx^2}.
\]

It was seen (here) that this equation was derived
from the original
\[
y = F(x+at) + f(x-at),
\]
where $F$ and $f$ were any arbitrary functions of $t$.

Another way of dealing with it is to transform it
by a change of variables into
\[
\frac{d^2y}{du · dv} = 0,
\]
where $u = x + at$, and $v = x - at$, leading to the same
general solution. If we consider a case in which
$F$ vanishes, then we have simply
\[
y = f(x-at);
\]
and this merely states that, at the time $t = 0$, $y$ is a
particular function of $x$, and may be looked upon as
denoting that the curve of the relation of $y$ to $x$ has
a particular shape. Then any change in the value
of $t$ is equivalent simply to an alteration in the origin
from which $x$ is reckoned. That is to say, it indicates
that, the form of the function being conserved, it is
propagated along the $x$ direction with a uniform
velocity $a$; so that whatever the value of the
ordinate $y$ at any particular time $t_0$ at any particular
point $x_0$, the same value of $y$ will appear at the subsequent
time $t_1$ at a point further along, the abscissa
of which is $x_0 + a(t_1 - t_0)$. In this case the simplified

equation represents the propagation of a wave (of any
form) at a uniform speed along the $x$ direction.

If the differential equation had been written
\[
m \frac{d^2y}{dt^2} = k\, \frac{d^2y}{dx^2},
\]
the solution would have been the same, but the
velocity of propagation would have had the value
\[
a = \sqrt{\frac{k}{m}}.
\]


You have now been personally conducted over the
frontiers into the enchanted land. And in order that
you may have a handy reference to the principal
results, the author, in bidding you farewell, begs to
present you with a passport in the shape of a convenient
collection of standard forms.
In the middle column are set down a number of the
functions which most commonly occur. The results
of differentiating them are set down on the left; the
results of integrating them are set down on the right.
May you find them useful!




\end{document}
